\documentclass[11pt,a4paper]{article}
\usepackage{amsmath,amssymb,amsfonts}
\usepackage{graphicx}
\usepackage{hyperref}
\usepackage{geometry}
\usepackage{float}
\usepackage{booktabs}
\usepackage{caption}
\usepackage{subcaption}
\usepackage{multirow}
\usepackage{bm}
\geometry{margin=1in}

\title{\textbf{MCMC Fitting of the $\Lambda$CDM Model using Observational Data}}
\author{Preet Dalal}
\date{\today}

\begin{document}
\maketitle

\section*{Overview}

This project demonstrates a complete Bayesian inference pipeline for the standard $\Lambda$CDM cosmological model using a combination of Cosmic Chronometers (CC), Baryon Acoustic Oscillations (BAO), and the UNION3 Type Ia supernova dataset. The objective is to reproduce standard literature by statistically constraining cosmological parameters by performing a full MCMC analysis with convergence diagnostics, goodness-of-fit tests, and posterior characterization.

This pipeline forms the numerical backbone of my cosmological data analysis and has been further extended to interacting dark energy models in modified $f(Q)$ gravity whose manuscript has been submitted to peer-reviewed journals for publications.

\section*{$\Lambda$CDM Model}

If we ignore the radiation density parameter $\Omega_{r0} \approx 8.4 \times 10^{-4}$ then our Hubble parameter is given by
\[
H(z) = H_0 \sqrt{\Omega_{m0}(1+z)^3 +  1 - \Omega_{m0}},
\]
where $\Omega_{m0}$ is the matter density parameter and $z$ is the redshift.

\section*{Datasets}

\subsection*{Cosmic Chronometer (CC)}

I used the latest 32 CC measurements of $H(z)$ spanning from $z = 0.070$ to $z = 1.965$. These are model-independent measurements derived from the differential ages of galaxies.

\subsection*{Baryon Acoustic Oscillations (BAO)}

The BAO measurements from SDSS DR7, SDSS DR12, and 6dFGS datasets provide distance constraints at multiple redshifts. 

\subsection*{Type Ia Supernovae (SNe Ia)}

The UNION3 Type Ia supernova compilation includes 580 supernovae with redshifts ranging from $z \approx 0$ to $z \approx 2.3$. 

\section*{MCMC Analysis}

The Bayesian inference is performed using the MCMC method with the Metropolis-Hastings algorithm.  The parameter space consists of $\Omega_{m0}$, $H_0$, and the absolute magnitude $M_B$ of Type Ia supernovae.

\subsection*{Prior Distributions}

Standard flat priors are adopted: 
\begin{itemize}
  \item $\Omega_{m0} \in [0.1, 0.5]$
  \item $H_0 \in [60, 80]$ km/s/Mpc
  \item $M_B \in [-19. 5, -19.0]$ mag
\end{itemize}

\subsection*{Convergence Diagnostics}

The Gelman-Rubin statistic $\hat{R}$ is computed for each parameter.  Convergence is achieved when $\hat{R} < 1.01$ for all parameters, indicating that the chains have properly explored the posterior distribution.

\section*{Results}

The MCMC analysis yields the following posterior constraints at $1\sigma$ confidence: 

\[
\Omega_{m0} = 0.297^{+0.012}_{-0.013}
\]

\[
H_0 = 70.2^{+1.8}_{-1.9} \text{ km/s/Mpc}
\]

\[
M_B = -19.11^{+0.05}_{-0.05} \text{ mag}
\]

These results are consistent with the latest Planck cosmic microwave background measurements and provide independent constraints on the cosmological parameters.

\section*{Summary}

\begin{itemize}
  \item Complete MCMC pipeline implemented for $\Lambda$CDM model
  \item Combined constraints from CC, BAO, and SNe Ia datasets
  \item All chains converged with $\hat{R} < 1.01$
  \item Results consistent with literature and CMB measurements
\end{itemize}

\end{document}