\documentclass[11pt,a4paper]{article}
\usepackage{amsmath,amssymb}
\usepackage{graphicx}
\usepackage{geometry}
\usepackage{hyperref}
\usepackage{caption}
\geometry{margin=1in}

\title{\textbf{Relativistic Precession in Schwarzschild and Reissner--Nordström Spacetimes}}
\author{Preet Dalal}
\date{}

\begin{document}
\maketitle

\vspace{-0.4cm}

\section*{Overview}

This project studies the relativistic perihelion precession of bound timelike orbits in the Schwarzschild (SCH) and Reissner--Nordström (RN) spacetimes.   
It is a purely theoretical and numerical investigation (not related to S2 star fitting) aimed at understanding how strong-field gravity and the charge parameter \(Q\) modify orbital dynamics beyond the Newtonian and first post-Newtonian (1PN) limits.  
The exact geodesic equations are integrated numerically and compared directly with analytic 1PN predictions. 


\section*{Spacetimes Considered}

Schwarzschild (SCH) spacetime:
\[
ds^2 = -\left(1-\frac{2M}{r}\right)dt^2
+ \left(1-\frac{2M}{r}\right)^{-1}dr^2
+ r^2 d\Omega^2 . 
\]
Reissner--Nordström (RN) spacetime:
\[
ds^2 = -\left(1-\frac{2M}{r}+\frac{Q^2}{r^2}\right)dt^2
+ \left(1-\frac{2M}{r}+\frac{Q^2}{r^2}\right)^{-1}dr^2
+ r^2 d\Omega^2 . 
\]
Here \(M\) is the mass of the black hole and \(Q\) is its charge parameter. 


\section*{Orbit Equations}

Introducing the reciprocal radial coordinate \(u = 1/r\) and using the conserved
specific angular momentum \(h\), the exact relativistic orbit equations take the form: 

\subsection*{Schwarzschild (SCH)}

\[
\frac{d^{2}u}{d\phi^{2}}
=
-\,u
+ 3Mu^{2}
+ \frac{M}{h^{2}} . 
\]

\subsection*{Reissner--Nordström (RN)}

\[
\frac{d^{2}u}{d\phi^{2}}
=
-\,u
+ 3Mu^{2}
+ \frac{M}{h^{2}}
- Q^{2}\!\left(\frac{u}{h^{2}} + 2u^{3}\right).
\]

These equations are integrated numerically to obtain the exact relativistic trajectories. 
The additional \(Q\)-dependent terms in the RN case are responsible for the qualitative
modifications of orbital dynamics and the possibility of negative precession. 


\section*{Circular Orbits and Stability}

Circular orbits are characterized using the effective potential \(V_{\rm eff}(r)\). Their stability is determined by
\[
\frac{d^2 V_{\rm eff}}{dr^2}
\begin{cases}
>0 & \text{Stable},\\[4pt]
=0 & \text{Marginally stable (ISCO)},\\[4pt]
<0 & \text{Unstable}.
\end{cases}
\]

For Schwarzschild spacetime,
\[
r_{\rm ISCO}^{\rm SCH} = 6M. 
\]

For Reissner--Nordström spacetime,
\[
r_{\rm ISCO}^{\rm RN}
=
3M + \sqrt{9M^2 - 8Q^2}.
\]

\begin{figure}[h]
\centering
\includegraphics[width=0.85\textwidth]{circular_sch. png}
\caption{Unstable, marginally stable (ISCO), and stable circular orbits in Schwarzschild spacetime obtained from the effective potential.}
\end{figure}

---

\section*{Relativistic Precession}

The perihelion precession is measured numerically as
\[
\Delta\phi = \phi_{n+1} - \phi_n - 2\pi,
\]
where \(\phi_n\) and \(\phi_{n+1}\) are two successive periapsis angles extracted from the orbit. 

At first post-Newtonian order, the analytic expressions are: 

For SCH:
\[
\Delta\phi_{\rm SCH}
=
\frac{6\pi M}{a(1-e^2)},
\]

For RN: 
\[
\Delta\phi_{\rm RN}
=
\frac{6\pi M}{a(1-e^2)}
-
\frac{3\pi Q^2}{a^2(1-e^2)^2},
\]

where \(a\) is the semi-major axis and \(e\) is the orbital eccentricity. 


\section*{Results}

\begin{figure}[h]
\centering
\includegraphics[width=0.9\textwidth]{precession_timelike.png}
\caption{Positive and negative perihelion precession in RN spacetime obtained from integrating orbit equations numerically.}
\end{figure}

\begin{figure}[h]
\centering
\begin{minipage}[c]{0.48\textwidth}
    \centering
    \includegraphics[width=\linewidth]{M_phi. png}
    \caption*{(a) $\Delta\phi$ vs $M$}
\end{minipage}
\hfill
\begin{minipage}[c]{0.48\textwidth}
    \centering
    \includegraphics[width=\linewidth]{a_phi.png}
    \caption*{(b) $\Delta\phi$ vs $a$}
\end{minipage}

\vspace{0.4cm}

\begin{minipage}[c]{0.48\textwidth}
    \centering
    \includegraphics[width=\linewidth]{e_phi.png}
    \caption*{(c) $\Delta\phi$ vs $e$}
\end{minipage}
\hfill
\begin{minipage}[c]{0.48\textwidth}
    \centering
    \includegraphics[width=\linewidth]{Q_phi.png}
    \caption*{(d) $\Delta\phi$ vs $Q/M$}
\end{minipage}

\caption{Variation of orbital precession with $M$, $a$, $e$, and $Q/M$.  Panels (a)--(c) show comparison with the Schwarzschild and 1PN predictions.  Panel (d) shows the effect of the charge parameter $Q$.}
\end{figure}

\end{document}